\documentclass{beamer}
\mode<presentation> {
\usetheme{Madrid}
% これで色変えれる %
\usecolortheme[RGB={0,100,125}]{structure}%this color is named itogreen.
\usepackage{zxjatype}
\usepackage[ipa]{zxjafont}
\setbeamertemplate{footline}[page number]
\setbeamertemplate{navigation symbols}{}
}
\usepackage{graphicx} 
\usepackage{booktabs} 
%\usepackage {tikz}
\usepackage{tkz-graph}
\usepackage{bm}
\GraphInit[vstyle = Shade]
\AtBeginSection[]{
    \begin{frame}
        \tableofcontents[currentsection]
    \end{frame}
}
\newcommand{\indep}{\mathop{\perp\!\!\!\perp}}
\tikzset{
  LabelStyle/.style = { rectangle, rounded corners, draw,
                        minimum width = 2em, fill = yellow!50,
                        text = red, font = \bfseries },
  VertexStyle/.append style = { inner sep=5pt,
                                font = \normalsize\bfseries},
  EdgeStyle/.append style = {->, bend left} }
\usetikzlibrary {positioning}

\newcommand{\ctext}[1]{\raise0.2ex\hbox{\textcircled{\scriptsize{#1}}}}
\title[Short title]{調査観察データの統計科学ゼミ-第7回-}
\author{Masami Chen Plio} % Your name
\institute[OU HUS Adachi Lab M1] 
{
Osaka University faculty of Human Sciences\\Adachi Lab M1\\
\medskip
}
\date{\today} 
\begin{document}
\begin{frame}
\titlepage
\end{frame}
\begin{frame}
\frametitle{担当章} 
\tableofcontents 
\end{frame}
\section{3.4 一般的な周辺パラメトリックモデルの推定}
\begin{frame}[fragile]{3.4 一般的な周辺パラメトリックモデルの推定}
\begin{itemize}
\setlength{\itemsep}{5mm}
\item うんこ
    \begin{itemize}
    \item うんち
    \end{itemize}
\end{itemize}
\end{frame}
%%%
\begin{frame}[fragile]{a}
\end{frame}
%%%

\begin{frame}[fragile]{a}
\end{frame}
%%%

\begin{frame}[fragile]{a}
\end{frame}
%%%

\begin{frame}[fragile]{a}
\end{frame}
%%%


\begin{frame}{お絵描き}
\end{frame}
\end{document}